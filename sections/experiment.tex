\section{Experiment}
\label{sec:experiment}

\subsection*{TopN Result}

\subsubsection*{Strict TopN Approach with Naive Adding}

In this experiment, we will use the most naive way to construct sentence embedding but try to see which N and which similarity is better. (Table~\ref{tab:different_similarity})

\begin{table}[htbp!]
    \centering
    \begin{tabular}{llccc}
    \toprule
        \multicolumn{2}{c}{Experiment}  & \multicolumn{3}{c}{WSI Metrics}           \\
    \midrule
        Model       & Similarity        & Fuzzy NMI     & Fuzzy B-Cubed  & Average  \\
    \midrule
        Top 1       & Cosine            & 2.81          & 50.22          & 11.89    \\
        Top 2       & Cosine            & 8.56          & 52.10          & 21.12    \\
        Top 3       & Cosine            & 7.15          & 42.87          & 17.51    \\
        Top 4       & Cosine            & 6.61          & 32.07          & 14.56    \\
        Top 5       & Cosine            & 5.92          & 24.80          & 12.11    \\
        Top 1       & Euclidean         & 3.63          & 47.78          & 13.17    \\
        Top 2       & Euclidean         & 7.87          & 46.06          & 19.04    \\
        Top 3       & Euclidean         & 6.97          & 40.36          & 16.77    \\
        Top 4       & Euclidean         & 6.46          & 34.25          & 14.87    \\
        Top 5       & Euclidean         & 6.15          & 28.33          & 13.20    \\
        Top 1       & Minkowski         & 3.63          & 47.78          & 13.17    \\
        Top 2       & Minkowski         & 7.87          & 46.06          & 19.04    \\
        Top 3       & Minkowski         & 6.97          & 40.36          & 16.77    \\
        Top 4       & Minkowski         & 6.46          & 34.25          & 14.87    \\
        Top 5       & Minkowski         & 6.15          & 28.33          & 13.20    \\
        Top 2 (BERT)& Cosine            & 6.74          & 46.78          & 17.76    \\
    \bottomrule
    \end{tabular}
\caption{Comparison of different Strict Top N model and different similarity with fastText embedding}
\label{tab:different_similarity}
\end{table}


\subsubsection*{Different TopN Approach with Different Sentence Embedding}

In this experiment, we test the combination of strict or generalized top N and using which sentence embedding. And shows the best result of each setting. (Table~\ref{tab:different_sentence_embedding})

\begin{table}[htbp!]
    \centering
    \begin{tabular}{llccc}
    \toprule
        \multicolumn{2}{c}{Experiment}          & \multicolumn{3}{c}{WSI Metrics}           \\
    \midrule
        Model              & Sentence Embedding & Fuzzy NMI     & Fuzzy B-Cubed  & Average  \\
    \midrule
        Strict Top 2       & Naive Adding       & 8.56          & 52.10          & 21.12    \\
        Generalized Top 2  & Naive Adding       & 8.35          & 52.40          & 20.92    \\
        Strict Top 2       & Normalized Adding  & \bf8.92       & \bf52.51       & \bf21.64 \\
        Strict Top 2       & Normalized Padding & 8.55          & 46.58          & 19.95    \\
    \bottomrule
    \end{tabular}
\caption{Comparison between different Top N model and different sentence embedding}
\label{tab:different_sentence_embedding}
\end{table}

\subsection*{BiLM Result}

\subsubsection*{Different Language Model with Different Clustering Target}

In this experiment, we will try to replace the original ELMo language model and see the effect of using different clustering target. (Table~\ref{tab:bilm_result})

\begin{table}[htbp!]
    \centering
    \begin{tabular}{llccc}
    \toprule
        \multicolumn{2}{c}{Experiment}   & \multicolumn{3}{c}{WSI Metrics}           \\
    \midrule
        Language Model     & Cluster     & Fuzzy NMI     & Fuzzy B-Cubed  & Average  \\
    \midrule
        ELMo               & one-hot     & 9.28          & 58.70          & 23.34    \\
        ELMo               & TF-IDF      & 11.06         & 57.72          & 25.27    \\
        ELMo               & BERT        & 2.65          & 54.34          & 12.00    \\
        ELMo               & GloVe       & 8.28          & 60.90          & 22.45    \\
        ELMo               & fastText    & 7.40          & 61.39          & 21.32    \\
        GloVe              & one-hot     & 8.66          & 59.21          & 22.64    \\
        GloVe              & TF-IDF      & 10.69         & 57.68          & 24.83    \\
        GloVe              & GloVe       & 8.08          & 60.88          & 22.18    \\
        GloVe              & fastText    & 6.49          & 61.08          & 19.91    \\
        fastText           & one-hot     & 8.78          & 58.59          & 22.68    \\
        fastText           & TF-IDF      & 10.82         & 57.34          & 24.90    \\
        fastText           & GloVe       & 7.79          & 60.52          & 21.72    \\
        fastText           & fastText    & 7.08          & 61.11          & 20.80    \\
    \bottomrule
    \end{tabular}
\caption{Comparison using different language model and using different clustering target}
\label{tab:bilm_result}
\end{table} 

\subsection*{Best Result}

Following will list the best result among our models and the best model in the 2013 competition. (Table~\ref{tab:best_result})

\begin{table}[htbp!]
    \centering
    \begin{tabular}{llccc}
    \toprule
        \multicolumn{2}{c}{Experiment}                       & \multicolumn{3}{c}{WSI Metrics}           \\
    \midrule
        Team                    & System                     & Fuzzy NMI     & Fuzzy B-Cubed  & Average  \\
    \midrule
        PKU NLP ForFun (Our)    & Strict Top 2 (Normalized)  & 8.92          & 52.51          & 21.64    \\
        Neural BiLM (2018)      & ELMo + TF-IDF              & 11.06         & 57.72          & 25.27    \\
        AI-KU (2013)            & Base                       & 4.50          & 35.10          & 12.57    \\
        Unimelb (2013)          & 50k                        & 3.90          & 44.10          & 13.11    \\
    \bottomrule
    \end{tabular}
\caption{Comparison between our model and the best model in the competition}
\label{tab:best_result}
\end{table}


\subsubsection*{BiLM, Substitute Vectors, Cluster}
Refer to figure 3, We use BiLM, Substitute Vectors, Cluster to extract multi-sense or meanings word which is ambiguation.
In our work, We choose ELMo, FastText, Glove as the BiLM model.
The ELMo pre-training model is from allennlp.
The fastText pre-training model is from   fasttext.cc.
The Glove pre-training model is from  Stanford. 

Before the clustering processing, we need a representatives method to extract substitute word meaning.
In this processing, we test for one-hot, TF-IDF, FastText, Glove, Bert, ELMo.
we used \emph{sklearn} for both one-hot, TF-IDF weighting and clustering.
The Bert model is running in a 12-head multi-transformer model. 
And the hidden state as the out params.

In backward substitute, we choose the random choose frequent is 4 words a group, and every sample sentence we random sampling for  20 times.
And the cluster num is 7.
We use agglomerative clustering (cosine distance, average linkage) and induce a fixed number of clusters.

From the experiment result, We found ELMo + TF-IDF is the best combination in our work.
Treating each representative as a document, TF-IDF reduces the weight of uninformative words shared by many representatives.

\subsubsection*{Test for Clustering Num}

\begin{table}[htbp!] % here top bottom page (! = remove further restrictions)
    \centering
    \begin{tabular}{llcccc}
    \toprule
    Cluster Number   & Fuzzy NMI     & Fuzzy B-Cubed  & Average    \\
    \midrule
    5                & 9.68          & \bf{58.88}     & 23.87      \\
    6                & 10.21         & 58.25          & 24.39      \\
    7                & 11.06         & 57.72          & 25.27      \\
    8                & 10.88         & 56.60          & 24.82      \\
    9                & \bf{11.58}    & 56.61          & \bf{25.60} \\
    10               & 10.95         & 55.94          & 24.75      \\
    \bottomrule
    \end{tabular}
\caption{BiLM in multiple Cluster Number (in \%)}
\label{tab:embedding}
\end{table}


And We also test for the cluster num.
And we found in this DataSet, 7 clusterings is the best params.
For more thought, when we defined the cluster num, It's a similarity to the top-N num.
It's decided by the data feature.